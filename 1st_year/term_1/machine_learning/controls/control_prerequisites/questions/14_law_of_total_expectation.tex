\section{Law of total expectation}
\subsection*{Conditional expectation and conditional variance}
\par
Suppose that $X$ and $Y$ are discrete r.v.-s, possibly dependent on each other (the same results hold for continuous r.v.-s too, but will assume for simplicity the first one case). Suppose that we fix $Y$ at the value $y$. This gives us a set of conditional probabilities $P(X = x| Y = y)$. This is called the conditional distribution of $X$, given that $Y = y$.

\begin{definition}{}{}
    Let $X$ and $Y$ be discrete random variables. The conditional probability function of $X$, given that $Y = y$, is:
    \[
        P(X = x| Y = y) = \dfrac{P(X = x \cap Y=y)}{P(Y=y)}  
    \]
\end{definition}

\begin{note}{}{}
    Notation:
    \useshortskip
    \[
        f_{X|Y}(x|y) = P(X = x| Y=y).    
    \]
\end{note}
\begin{definition}{(Conditional expectation)}{}
    Let $X$ and $Y$ be discrete random variables. The conditional expectation of $X$, given that $Y = y$, is:
    \[
        \mu_{X|Y=y}  = \E [X | Y = y] = \sum\limits_{x} xf_{X|Y}(x|y).
    \] 
\end{definition}
\begin{note}{}{}
    Intuition: $E[X | Y = y]$ is the mean value of $X$, when $Y$ is fixed at $y$.
\end{note}

\begin{note}{}{}
    Conditional expectation, $\E (X|Y)$, is a random variable with randomness inherited from $Y$, not $X$.
\end{note}
The conditional variance is similar to the conditional expectation:
\begin{itemize}
    \item $\var (X| Y = y)$ is the variance of $X$, when $Y$ is fixed at the value $Y = y$;
    \item $\var (X|Y)$ is a random variable, giving the variance of $X$ when $Y$ is fixed at a value to be selected randomly.
\end{itemize}
\begin{definition}{(Conditional variance)}{}
    Let $X$ and $Y$ be random variables. The conditional variance of $X$, given $Y$, is given by:
    \[
        \begin{array}{c}
            \var (X|Y) = \E (X^2 | Y) - \left\{\E (X|Y)\right\}^2 = \\ = \E \left\{\left(X - \mu_{X|Y}\right)^2 | Y\right\}              
        \end{array}
    \]
\end{definition}

\begin{note}{}{}
    Like expectation, $\var \left(X | Y =y \right)$ is a number depending on $y$, while $\var (X|Y)$ is a random variable with randomness inherited from $Y$.
\end{note}

\subsection*{Law of total expectation}

\[
    \E X = \E_Y \left[X | Y\right],
\]
where $\E_Y$ is denoted by expectation over $Y$, i.e. the expectation is computed over the distribution of the random variable $Y$.
\begin{note}{}{}
    The law of total expectation says that the total average is the average of case-by-case averages.
\end{note}