\section{Law of total variance}

\begin{theorema}{(Law of total variance)}{}
    \useshortskip
    \[
        \var  X = \E_Y\left[\var (X | Y)\right] + \var_Y \left(\E[X|Y]\right),
    \]
    where $\E_Y$ and $\var_Y$ denote expectation over $Y$ and variance over $Y$.
\end{theorema}
\par The variance is computed over the distribution of the r.v. $Y$.
\par
Let's rationale about the terms:
\begin{thesis}{}{}
    What is $\E_Y \left[\var \left(X | Y\right)\right]$?
\end{thesis}
\par 
Is the average of the variance of $X$ over all possible values of the random variable $Y$. In other words: take the variance of $X$ in each conditional space of $Y = y$. Then, take the average of the variances. This is called the average within-sample variance.

\begin{thesis}{}{}
    What is $\var_Y (\E[X|Y])$?
\end{thesis}
\par 
Note that the first term $\E_Y \left[\var \left(X | Y\right)\right]$, only considers the average of the variances of $X | Y$. That term does not take into account the movement of the mean itself, just the variation about each, possibly varying, mean.

\par
If we treat each $Y = y$ as a separate ``treatment'', then the first term is measuring the average within-sample variance, while the second is measuring the between-sample variance.